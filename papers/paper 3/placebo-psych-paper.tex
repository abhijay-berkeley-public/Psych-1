% Originally created by Nicholas Diaz (nsdiaz@uc.cl)
% Modifications by Vel (vel@latextemplates.com)
% Downloaded from LateX Templates (https://www.latextemplates.com/template/thin-sectioned-essay)
% Simplified and further modified by Eugene Nakamoto (ynakamot@u.rochester.edu)
% License:
% CC BY-NC-SA 3.0 (http://creativecommons.org/licenses/by-nc-sa/3.0/)

\documentclass[a4paper, 10pt]{article}
\usepackage{geometry}
\geometry{margin=1.0in}
\usepackage{mathpazo}
\usepackage[T1]{fontenc}
\usepackage{setspace}
\doublespacing
\usepackage{type1cm}
\usepackage{lettrine}
\usepackage{lipsum}

\makeatletter
\renewcommand{\maketitle}{
\begin{flushright}
{\LARGE\@title}

\vspace{0pt}

{\large\@author}
\\\@date

\vspace{0pt}
\end{flushright}
}

\title{\textbf{The Placebo Effect}
\\ Psychology 1}

\author{\textsc{Abhijay Bhatnagar}}

\date{\today}

\begin{document}

\maketitle
\lettrine{O}{ne} of the most well-known phenomena in psychology is placebo effect: the seemingly miraculous phenomena in which the consumption of a conventionally non-medicinal substance (traditionally a sugar pill) has observable curative effects. The effect is commonly understood to be an extension of the body's ability to natural repair itself. In this assignment, I decided to compare how different types of scientific media explore and discuss the placebo effect and the effect expectations
 can have on the body's natural healing capacity. For my peer-reviewed primary research article, I analyze an article published in \textit{Health Psychology} titled "Harnessing the Placebo Effect: Exploring the Influence of Physician Characteristics on Placebo Response" (Howe, Goyer, \& Crum, 2017). This article focused on a particular aspect of placebo effect, specifically the influence of patient expectations as impacted by the medical provider's behavior. Similarly, for my popular press article I found an article in \textit{Time
 Magazine} titled "Placebo's New Power". It also covers placebo medication and patient expectations, but gives a much broader over, weaving in personal stories, multiple studies, and other relevant information about the practice (Sifferlin, 2018). 


 In their research, Howe, Goyer, and Crum look for answers to "the puzzle of when placebo/nocebo effects are most likely to occur" (Howe, Goyer, \& Crum, 2017). From past studies cited in the article, they know that patient expectations have been consistently observed to be the primary driver of the placebo/nocebo (the negative health outcome counterpart) effect; however, much of the variability of the placebo effect is not yet fully understood. Building off of that foundation, the researchers hypothesize that patients' expectations
 are heavily influenced by the social environment in which they are determined, impacting the likelihood, intensity, and nature of their placebo outcomes (Howe, Goyer, \& Crum, 2017). To concretely test this, in this study they set up an experiment that tests a placebo cream applied to mild allergic reactions caused by a histamine skin prick test (SPT), conducted in a controlled environment with a trained health care provider. In this scenario, they are manipulating the provider's behavior (i.e., their warmth and apparent
 competence) and conveyed expectations about the cream's behavior, and are seeking to answers what the effects of those variables are on the physiological outcome (Howe, Goyer, \& Crum, 2017). They hypothesize that a seemingly competent and friendly provider who tells the patient the cream is beneficial will increase the strength of the patients positive expectations about the treatment; similarly, a seemingly competent and friendly provider who tells the patient the cream will exacerbate their condition will lead to an increase
 of negative expectations. However, an unlikeable and seemingly incompetent provider will not significantly influence the patient's expectations regardless of the stated effect of the cream. Finally, they consider any hybrid stage of the provider's behavior may produce less pronounced effects on expectations (Howe, Goyer, \& Crum, 2017).

They initially recruited 164 healthy adult participants of unspecified gender, race, and age under the pretense of a food related study. (Later in the study, they recruit an additional 30 participants for additional control subjects.) Then they proceeded with the experiment by telling the participants they had to undergo a health evaluation before the food study (Howe, Goyer, \& Crum, 2017). The researchers take them to the controlled room, where they meet the health provider. This provider is trained to act in one of four ways: warm and
 competent, warm and incompetent, cold and competent, and cold and incompetent. Similarly, the room is decorated to reflect the desired atmosphere. The provider then collects basic vitals and health information from the subject, and proceeds with the SPT. After the histamine reaction is induced on the patient, the provider then applies a placebo lotion on the affect area, while telling the subject to expect either positive or negative reactions from the lotion (some control subjects are not told
 what to expect). Afterwards, the affected area is traced every 3 minutes for 15 minutes (the reaction peaks at 6 and subsides at 15), and the tracing are sent for measurement to a research assistant blind to testing conditions (to control for confirmation bias). These will be compared with the results of the other testing conditions and the controls to measure the placebo's physiological effects. Finally, to confirm the testing conditions and assess perceived placebo effects, the subjects are surveyed for the itchiness
 of their skin, their level of comfort and anxiety, their experience, and their opinion on the provider's warmth and competence (Howe, Goyer, \& Crum, 2017).

The results of the study fell mostly along the lines of the researcher's hypothesis. Through the participant surveys, they confirmed their manipulation of warmth and competence created the desired perceptions (Howe, Goyer, \& Crum, 2017). In each analysis, they controlled for the subject's race and gender, and the relative rate of histamine reaction. The researcher's began by giving the simpler group by group analysis. For subjects served by warm and friendly providers, those given positive expectations about the cream were
 observed to have healed quicker than those given negative expectations (with the no expectation control group in between both results). For subjects served by cold and unfriendly providers, there were no differences in healing rates regardless of expectations. The subjects served by hybrid-behavior had intermediate results compared to the previous two groups. The researchers also did more complex multi group analysis to determine how much warmth and competence moderated expectations, and they determined
 that provider warm and competence strengthened positive expectations, but did not strengthen negative expectations (Howe, Goyer, \& Crum, 2017).

	To conclude, the researchers observed that a likeable and competent doctor only strengthens positive expectations (contrary to their hypothesis of strengthening both expectations), and the social context can both increase and decrease the potency of the placebo effect--a result that disagrees with one part of their original hypothesis (Howe, Goyer, \& Crum, 2017). However, the implications of this are still important. The researchers have helped demystify some of the factors in placebo effect variability, and they have introduced
 a new methodology for placebo testing (applying inactive cream after SPT). The researcher's also pointed out that these results are applicable in general medical practice, and that it is critical to consider the social environment of treatment, and how non-placebo medicine may itself be more effective when provided with warmth and competence (Howe, Goyer, \& Crum, 2017).

Stepping back a few feet from the specificity of the peer-reviewed article, in the popular press article "Placebo's new Power", Sifferlin gives us a much larger overview of Placebo medication and the role of its healing effects within the country (Sifferlin, 2018). The article itself does not lead with any specific thesis or particular claim, yet instead weaves many different narrative and factual elements together to paint a picture of the phenomena. Nevertheless, despite each topic being split into many discrete
 parts told in a nonlinear order, I've identified four main claims within the, in no order of importance. 

First, the author is making a general claim that placebo medication is  counterintuitive; however, scientific studies are continually reaffirming and testing its real medicinal value (Sifferlin, 2018). Through descriptive language and personal statements, she makes it fairly clear that she recognizes the counterintuitive nature of placebos. That being said, she only explicitly provides evidence of this confusion through the voice of others, such as when she quoted a placebo patient saying "I didn't have a clue what
 was going on... I still don't." after being successfully treated for irritable bowel syndrome (Sifferlin, 2018, p. 3). She quickly connects this counterintuitive phenomena to various studies that validate the efficacy of placebo treatments, mostly focusing on the work of Ted Kaptchuk and the 'honest placebos'  (Sifferlin, 2018, p. 3).  (Later in the artcle, she even references the exact study of the peer-reviewed paper I selected.) The specific details of each study vary, but the overall trend is almost always an increase in healing outcomes
 when compared to patients not taking medicine, and in some cases comparable or even better outcomes compared to the actual medicine  (Sifferlin, 2018). 

She interweaves these seemingly miraculous results with a balancing claim that even if there are many studies being done, this phenomena is still not completely understood and conflicting viewpoints exist  (Sifferlin, 2018). As evidence, she points to the opinion of experts, and more specifically experts who disagree. In the explanation of how placebo works, she briefly explains two contrary theories held experts, one being pavlovian conditioning to treatment itself, the other citing the power of positive
 thoughts (Sifferlin, 2018). In a section discussing how research on the placebo effect should translate to practice, she outlines many dissenting opinions that are largely skeptical and wary of placebo medication and their corresponding studies (Sifferlin, 2018).

Next, throughout the pieces she is building a claim that despite placebos not being completely understood, they are definitely valued (in America) (Sifferlin, 2018). She starts off the entire piece with a personal story told in narrative form about a women who is unable to find medical relief for her excruciating irritable bowel syndrome, until she begins a placebo treatment (Sifferlin, 2018). This demonstrates personal value on an individual scale. The author then mentions several of the studies being done in elite institutions on
 the topic, receiving millions of dollars in funding, demonstrating evidence of institutional importance. She gives examples of independent entrepreneurs finding markets for selling placebos, demonstrating business value, and finally she even gives us a brief history on the historical importance of the phenomena, dating back to WWII and even Thomas Jefferson (Sifferlin, 2018).
	
	The value of placebo leads directly to her concluding claim. Through small snippets throughout the piece (and in the subtitle of article itself), she makes the claim that modern medicine in America lacks a personal touch and empathy, much to its own detriment, and that placebos may be a step in the right direction (Sifferlin, 2018). She backs up her claims on the state of modern medicine through both generalized statements accurately reflecting unfortunate truths about "higher medical bills and less and less face
 time with doctors," and through statistics that show American confidence in the healthcare system has gone down nearly \%40 in the last half century (Sifferlin, 2018). Building off the work of Crum (in the same peer-reviewed article), the author then emphasizes how placebos (and other alternative medicines) are directly tied to healthy, healing social environments (Sifferlin, 2018). She concludes her piece on this note, that the existence and power of placebos is evidence that personal caregiving cannot be ignored (Sifferlin, 2018).

In comparison, these two articles both touch on the topic of the placebo effect from different perspectives. The peer-reviewed article studies in great detail through experimentation the specifics of how the phenomena works, while the popular press explores how the placebo effect in the context of individual people, and the collective consideration of studies across the country, including the peer-reviewed article. Stylistically, the two articles also greatly differ. The peer-reviewed article is
 voiced without much emotion in a matter-of-fact manner, while the popular press piece uses emotion to strongly emphasize its statements. The peer-reviewed article is also organized into discrete sections, each with separate specific purposes, while the popular press article is one singular piece without clear separations or sections. Finally, the peer-reviewed article makes very few claims without citing a source, and makes even fewer extrapolations without further evidence. On the other hand, the
 popular press piece makes use of very few citations, and often uses their referenced claims to then make further extrapolations grounded almost entirely in opinion. That being said, the two paper styles are also similar in some ways. For instance, they both use past studies on the topic to form the foundations of their articles. They also both connect their concluding claims back to modern medicine and the potential for improvements.
	
	In conclusion, if you are aiming for breadth of knowledge, the popular press article incorporates information from the results of many different studies with plenty of background information and additional relevant topics, giving you a brief but multifaceted understanding of the topic. On the hand, if you are trying to more strictly understand narrow and specific details about a topic, or verify the data supporting generalized conclusions, the peer-reviewed article is more appropriate. 

\newpage
\section*{References}

Howe, L. C., Goyer, J. P., \& Crum, A. J. (2017). Harnessing the placebo effect: Exploring the influence of 

physician characteristics on placebo response. \textit{Health Psychology}, 36(11), 1074-1082. 

doi:http://dx.doi.org/10.1037/hea0000499  

\

\noindent Sifferlin, A. (2018). Placebo's New Power. \textit{Time}, 192(9/10), 64-69. Retrieved November 12, 2018, 

from http://search.ebscohost.com/login.aspx?direct=true\&db=a9h\&AN=131380547\&site=ehost-live

\end{document}