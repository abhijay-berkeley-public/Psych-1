% Originally created by Nicholas Diaz (nsdiaz@uc.cl)
% Modifications by Vel (vel@latextemplates.com)
% Downloaded from LateX Templates (https://www.latextemplates.com/template/thin-sectioned-essay)
% Simplified and further modified by Eugene Nakamoto (ynakamot@u.rochester.edu)
% License:
% CC BY-NC-SA 3.0 (http://creativecommons.org/licenses/by-nc-sa/3.0/)

\documentclass[a4paper, 10pt]{article}
\usepackage{geometry}
\geometry{margin=1.0in}
\usepackage{mathpazo}
\usepackage[T1]{fontenc}
\usepackage{setspace}
\doublespacing
\usepackage{type1cm}
\usepackage{lettrine}
\usepackage{lipsum}

\makeatletter
\renewcommand{\maketitle}{
\begin{flushright}
{\LARGE\@title}

\vspace{0pt}

{\large\@author}
\\\@date

\vspace{0pt}
\end{flushright}
}

\title{\textbf{Developing Child Paper}
\\ Psychology 1}

\author{\textsc{Abhijay Bhatnagar}}

\date{\today}

\begin{document}

\maketitle
\lettrine{T}{he} subject for childhood development analysis is Anika Bhatnagar, an 10 year-old female living of Indian descent, standing at a height of 4'10'' and approximately median weight for her age, height, and sex. She currently resides in Massachusetts, living in a small suburban neighborhood just outside of Boston. Her town is rated extremely high in safety and comfortable, with well regarded school systems, community programs, and extracurricular opportunities. Anika is accompanied by a nurturing and providing family, consisting of her mother and father, as well as two older brothers (of 19 and 22 years of age). As one of these brothers, I had the opportunity this past week to call the subject to better understand her past and current developments, as well as administer a small test to test her cognitive development.

	I began by inquiring into the subject's development of their individual sense of identity, with a focus modeled on the theory of Erikson due to her young age (Kalat, 2016, p. 163). Anika described her past to have consisted of typical issues for peers in similar socioeconomic situations. As an infant in the first stage of development, she was treated with love and support from her family with plenty of nurturing contact especially from sleeping in the parents' bed, enabling a strong foundation of trust. In her toddler years, she balanced increasing freedoms within the house with the well-meaning support of her family outside of the house. A combination that developed a strong sense of confidence within the family but nervousness and shyness when handling new things outside of the home. As a preschool-aged child her time was spent between kids of her own age, and her two elder, and adventurous brothers. Her sense of self-confidence and initiative developed as the lessons learned from pretend-playing with her elder siblings translated into more purposeful interactions with her peers. Now, as a school-aged child, she is learning to deal with what Erikson calls industry vs. inferiority, or the 4th stage of development (Kalat, 2016, p. 163). In the 6th grade, she's slowly learning where she stands academically, artistically, and socially compared to the outside world. In her own opinion, she's learning to prioritize her strengths over her weaknesses, and is overall straddling the contentment/inferiority complex line. 
	
	In addition to developing her sense of identity, her and I spoke at length about the development of her cognition, framed through Piaget's theory of cognitive development (Kalat, 2016, p. 152). From her youngest ages, we observed development during her Sensorimotor Stage through her interactions with the objects in the world around her, observing growth to correspond to diminished responses to 'peek-a-boo' style games, indicating her growing sense of object permanence (Kalat, 2016, p. 152). In this period, she developed an understanding of basic environments and initial schemata for how the world works. As she aged into the Preoperational Stage, she began to develop her usage of words and symbols, as well as limited but increased logical thinking (Kalat, 2016, p. 155). She began to learn Hindi and English, and struggled to distinguish when to use which language. One particular early schema she developed was associated almost all technology as 'iPods' -- phones, tablets, e-readers, etc. As she aged past 6-7 years old, her ability to distinguish between these ambiguities increased. She did not observe significant difficulty in object constancy past a very early age, and similarly quickly overcome early egocentrism. Currently, I would categorize her to be in the Concrete Operational Stage (Kalat, 2016, p. 159). She has developed her intellectual inquiries past the basics of fact gathering and into logical reasoning, but it has not quite reached high levels of abstraction. In school and in playgrounds, she has observed an increase in her own ability to perceive situations from her peers perspectives. Overall, she has become self-aware of her own emotional stream of consciousness -- observed through a diary she actively updates -- and is slowly heading towards the abstraction and deductive reasoning of the Formal Operational Stage (Kalat, 2016, p. 159).
	
	Building off Piaget's theories, the two of us also touched upon her moral development through the lens of Kohlberg's theory. As a younger child, it was only relevant to look at the first two levels / four stages of moral development in Kohlberg's theory, the Preconventional and Conventional levels (McLeod, 2013). More specifically, Anika seemed to progress through the First Stage (Obedience/Punishment) fairly normally during infancy, where her actions were less dictated by a moral compass, and more so by the scolding and punishment of her parents. As she entered the pre-school age and Second Stage, Anika became more aware of incentives laid out by her mother to perform as she would like, e.g. getting a snack after school when she put away her book bags and boots in an orderly fashion (McLeod, 2013). As she aged and started attending school, she entered the Conventional Level and Third Stage (Interpersonal Relationships) of moral development as she became increasingly conscious of how others perceived her and her actions (McLeod, 2013). This meant an increase in proactive behavior that would influence others opinions, such as keeping her room clean unprompted and speaking with polite etiquette, as opposed to responsive behavior to parental scolding/rewards. Additionally, currently she is a step above that, firmly in Kohlberg's 4th stage (Social Order), where she proactively acts to uphold social order (McLeod, 2013). In practice this materializes in 'good' decision making when orders are not there to observe, as well as reminding (and scolding) others who are behaving against social good. While Kohlberg's model of moral development is not considered by consensus perfect, from my conversations with Anika, the model seems to have matched fairly accurately as far as vague age ranges are considered.
	
	One of the final topics we discussed (before the experiment) was attachment in the context of Ainsworth's Theory of Attachment (Kalat, 2016, p. 164). As an infant, Anika definitely felt a strong attachment towards her mother, expressing comfort in her presence and interest in her opinion on her explorations around her play pen. During social gatherings with family friends who in her opinion were unfamiliar strangers, she would gravitate towards her mother (and father), but ultimately felt free to explore the room in their presence. In a hypothetical conversation detailing the experiment Ainsworth conducted, Anika claimed responses most similar to that of a Securely Attached child (Kalat, 2016, p. 164). While there is some grey area with regards to her verbalizing her responses as opposed to having had the experiment conducted on her (ten years ago), as a person present in her upbringing I agreed with her sentiments, and firmly believe her to be Securely Attached.
	
	Lastly, after going over much of her life and her development, her and I conducted an experiment (with the help of my father). Specifically, we conducted a Piagetian Conservation Task, which involves testing a child's ability to accurately recognized conserved physical properties (Kalat, 2016, p. 157). The typical procedure involves measuring presenting observably obvious equal quantities of a physical object, verifying the infant can recognize the initial state as equal, slightly modifying the objects such that the visual form is distorted but the original property is conserved, and testing if the child can verify and justify the original property is conserved (Kalat, 2016, p. 158). As a 10 year old, Anika is expected to pass and justify all of the basic Piagetian tasks, and so we hope to observe that she is able to verify the conservation and provide logical justifications for it (Kalat, 2016, p. 161). 

We proceeded with the most appropriate task of conservation of volume. To conduct the experiment, we equally filled up two clear glasses of water and placed equal volume balls of play-doh inside the cups. When asked if the water rose equally from the two balls, she confirmed it had, which suggests she has an internal schema for water displacement. We then removed one ball and deformed its shape. We then asked her if the mass of clay would still raise it an equal amount if placed back in the cup. She responded in the affirmative, suggesting that we could always reshape the mass back into a ball, therefore its volume didn't change. This not only tells us that she was able to pass and meet expectations, but that she was also able to assimilate her existing schema to handle the new situation (Kalat, 2016, p. 152). This corroborates our previous observation that she is likely near the end of the Concrete Observational Stage.

To give future advice to Anika, I would draw inspiration from the work of Vygotsky. Vygotsky's Social Development Theory differs from Piaget in that it focuses on how the social environment can help influence development, and how competent elders can accelerate growth into latter "zones of development" (Kalat, 2016, p. 160). With those concepts in mind I would encourage her, in this zone of proximal development, to spend more time around her older siblings and family and to join extracurricular activities with positive role models, such as her theatre group and choir. She is developing well for a child of her age, but as she grows older and is soon to hit the issues of abstract thought in Formal Observational Stage, the more advice and lessons she can learn from her peers, the better.
All together, this conversation was very illustrative in learning both the past and current development of my sister, Anika, and helped place much of her behaviors within the current psychological theories of development.

\newpage
\section*{References}

Kalat, J. W. (2016). \textit{Introduction to psychology}. Boston, MA: Cengage Learning.
\\
McLeod, S. A. (2013). \textit{Kohlberg}. Retrieved from https://www.simplypsychology.org/kohlberg.html

\end{document}